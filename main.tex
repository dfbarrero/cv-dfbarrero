%% start of file `main.tex'.
%% Copyright 2014 Francois Mouton (moutonf@gmail.com).
%
% This template is adapted from the work performed by Xavier Danaux (xdanaux@gmail.com).
% This template further extends the functionality by integrating the moderntimeline package.
% This template also includes custom Biblatex style to print bibliography items with the moderntimeline package.
%
% This work may be distributed and/or modified under the
% conditions of the LaTeX Project Public License version 1.3c,
% available at http://www.latex-project.org/lppl/.


\documentclass[11pt,a4paper,sans]{moderncv}        % possible options include font size ('10pt', '11pt' and '12pt'), paper size ('a4paper', 'letterpaper', 'a5paper', 'legalpaper', 'executivepaper' and 'landscape') and font family ('sans' and 'roman')

% moderncv themes
\moderncvstyle{classic}                             % Only the 'classic' style is fully functional with the modifications made. The other options, 'casual' (default), 'oldstyle' and 'banking' has minor typesetting problems with the current modifications.
\moderncvcolor{blue}                               % color options 'blue' (default), 'orange', 'green', 'red', 'purple', 'grey' and 'black'
%\renewcommand{\familydefault}{\sfdefault}         % to set the default font; use '\sfdefault' for the default sans serif font, '\rmdefault' for the default roman one, or any tex font name

% character encoding
\usepackage[utf8]{inputenc}                       % if you are not using xelatex ou lualatex, replace by the encoding you are using

% adjust the page margins
\usepackage[scale=0.75]{geometry}
%\setlength{\hintscolumnwidth}{3cm}                % if you want to change the width of the column of the timeline
%\setlength{\makecvtitlenamewidth}{10cm}           % for the 'classic' style, if you want to force the width allocated to your name and avoid line breaks. Be careful though, the length is normally calculated to avoid any overlap with your personal info; use this at your own typographical risks.

%-------------------Inlcuding pdfpages package-------------------------------------------------------------

\usepackage{pdfpages/pdfpages}

%-------------------Including moderntimeline package-------------------------------------------------------

\usepackage{moderntimeline/moderntimeline}

\tlmaxdates{2004}{2018}                             % Set the scale of the timeline. \tlmaxdates{startDate}{endDate}

%-------------------Including xpatch package---------------------------------------------------------------

\usepackage{xpatch/xpatch}

%-------------------Including Biblatex package-------------------------------------------------------------

\usepackage[url=false,
    backend=biber,                                  % This can be set to either biber or bibtex. If references are missing just change back and forth between biber and bibtex..
    style=authoryear,
    doi=false,  
    isbn=false,
    backref=false,
    dashed=false,                                   % Do not add a dash out authors for subsequent articles with the same authors.
    maxnames=99,                                    % Amount of authors to include before abbreviating.
    sorting=ydnt]{biblatex}                         % Sorting in reverse order

\addbibresource{cvreferences.bib}                   % Include your bibtex file here. Format: fileName.bib
\addbibresource{cvprojects.bib} % Include your bibtex projects file here. Format: fileName.bib
\addbibresource{cvbchthesis.bib} % Include your bibtex projects file here. Format: fileName.bib
\addbibresource{cvcourses.bib} % Include your bibtex projects file here. Format: fileName.bib
\addbibresource{cvgrants.bib} % Include your bibtex projects file here. Format: fileName.bib
\addbibresource{cvothers.bib} % Include your bibtex projects file here. Format: fileName.bib


\input{biblatex_modifications/standard_modification.tex}        % Modifying the default standard.tex style of Biblatex. This modification is performed to include the moderntimeline package.
\input{biblatex_modifications/project_modification.tex}        % Modifying the default standard.tex style of Biblatex. This modification is performed to include the moderntimeline package.
\input{biblatex_modifications/bchthesis_modification.tex}        % Modifying the default standard.tex style of Biblatex. This modification is performed to include the moderntimeline package.
\input{biblatex_modifications/course_modification.tex}        % Modifying the default standard.tex style of Biblatex. This modification is performed to include the moderntimeline package.
\input{biblatex_modifications/grant_modification.tex}        % Modifying the default standard.tex style of Biblatex. This modification is performed to include the moderntimeline package.
\input{biblatex_modifications/other_modification.tex}        % Modifying the default standard.tex style of Biblatex. This modification is performed to include the moderntimeline package.

%-------------------Defining a CV Reference column style and a CV reference entry block-------------------

%-------------------Defining a CV Reference column style and a CV reference entry block-------------------

% Adapted from the solution provided in: http://tex.stackexchange.com/questions/34881/references-section-in-a-cv
% usage: \cvreference{name}{address line 1}{address line 2}{address line 3}{address line 4}{e-mail address}{phone number}{mobile phone number}
% Everything but the name is optional
% If \addresssymbol, \emailsymbol or \phonesymbol are specified, they will be used.
% (Per default, \addresssymbol isn't specified, the other two are specified.)
% If you don't like the symbols, remove them from the following code, including the tilde ~ (e.g. \phonesymbol~).

\newcommand{\cvreferencecolumn}[2]{%
  \cvitem[0.75em]{}{%
    \begin{minipage}[t]{\listdoubleitemmaincolumnwidth}#1\end{minipage}%
    \hfill%
    \begin{minipage}[t]{\listdoubleitemmaincolumnwidth}#2\end{minipage}%
    }%
}

\newcommand{\cvreference}[8]{%
    \textbf{#1}\newline% Name
    \ifthenelse{\equal{#2}{}}{}{\addresssymbol~#2\newline}%
    \ifthenelse{\equal{#3}{}}{}{#3\newline}%
    \ifthenelse{\equal{#4}{}}{}{#4\newline}%
    \ifthenelse{\equal{#5}{}}{}{#5\newline}%
    \ifthenelse{\equal{#6}{}}{}{\emailsymbol~\texttt{\href{mailto:#6}{\nolinkurl{#6}}}\newline}%
    \ifthenelse{\equal{#7}{}}{}{\phonesymbol~#7\newline}
    \ifthenelse{\equal{#8}{}}{}{\mobilephonesymbol~#8}}

%-------------------Personal Data for CV title-----------------------------------------------------------
% Example:
\name{David}{F. Barrero}
\title{Curriculum Vit\ae}                               % optional, remove / comment the line if not wanted
\address{Departamento de Automática}{Escuela Politécnica Superior}{Alcalá de Henares, Spain}% optional, remove / comment the line if not wanted; the "postcode city" and and "country" arguments can be omitted or provided empty
%\phone[mobile]{+1~(234)~567~890}                   % optional, remove / comment the line if not wanted
\phone[fixed]{+34~91~885~6920}                    % optional, remove / comment the line if not wanted
%\phone[fax]{+3~(456)~789~012}                      % optional, remove / comment the line if not wanted
%\email{john@doe.org}                               % optional, remove / comment the line if not wanted
\homepage{https://atc1.aut.uah.es/\~{ }david}                         % optional, remove / comment the line if not wanted
\extrainfo{Profesor Titular de Universidad}                 % optional, remove / comment the line if not wanted
\photo[64pt][0.4pt]{images/dfbarrero.jpg}                       % optional, remove / comment the line if not wanted; '64pt' is the height the picture must be resized to, 0.4pt is the thickness of the frame around it (put it to 0pt for no frame) and 'picture' is the name of the picture file stored
%\quote{Some quote}                                 % optional, remove / comment the line if not wanted

%-------------------------------------------------------------------------------------------------------
%   Content
%-------------------------------------------------------------------------------------------------------
\begin{document}

%-------------------Resume------------------------------------------------------------------------------

\makecvtitle

%-------------------Education Section-------------------------------------------------------------------

\section{Education}

% For a date range: (To indicate 'up to present', set EndYear to 0)
% Format:  \tlcventry{StartYear}{EndYear}{Degree}{Institution}{City}{\textit{Grade}}{Description}  % Arguments 3 (Degree) to 6 (Grade) can be left empty. 
% Example: \tlcventry{2012}{0}{BSc Computer Science}{University of MyCity}{MyCity}{}{Also completed several random courses}

%\tlcventry{2004}{0}{Telecommunications Engineering}{Universidad de Alcalá}{}{}{}

% For a single year:
% Format:  \tldatecventry{StartYear}{Degree}{Institution}{City}{\textit{Grade}}{Description}
% Example: \tldatecventry{2008}{Senior Certificate}{High School MyCity}{MyCity}{\textit{80\%}}{Passed with distinction}

\tldatecventry{2011}{PhD in Computer Science}{Universidad de Alcalá}{}{}{}
\tldatecventry{2007}{Advanced Studies Certificate}{Universidad de Alcalá}{}{}{}
\tldatecventry{2004}{Telecommunications Engineering}{Universidad de Alcalá}{}{}{}

%-------------------PhD Thesis Section------------------------------------------------------------------

\section{PhD thesis}

% Format:  \cvitem{Section Name}{Description}
% Example: \cvitem{title}{\emph{The title of my PhD goes here}}
% Example: \cvitem{supervisors}{My supervisors' names go here}
% Example: \cvitem{description}{Short thesis abstract}

\cvitem{Title}{\emph{Reliability of Performance Measures in Tree-Based Genetic Programming: A study on Koza’s Computational Effort}}
%\cvitem{Supervisors}{María Dolores R. Moreno, David Camacho}
\cvitem{Calification}{Excellent \textit{cum laudem} (10). European mention.}
%\cvitem{description}{Short thesis abstract}


%-------------------Interests Section-------------------------------------------------------------------

%\section{Interests}

% Format:  \cvitem{Hobby}{Description}
% Example: \cvitem{Gaming}{Computer Games}
% Example: \cvitem{Sport}{Golf, Tennis}

%\cvitem{Gaming}{Computer Games}
%\cvitem{Sport}{Golf, Tennis}

%-------------------Experience Section------------------------------------------------------------------

\section{Experience}

%-------------------Vocational Experience---------------------------------------------------------------

\subsection{Positions}

% Format: \tlcventry{StartYear}{EndYear}{Job title}{Employer}{City}{Country (optional)}{General description no longer than 1--2 lines.\newline{}%
% Example:
% \tlcventry{2008}{2011}{System Administrator}{Simple Solutions}{MyCity}{}{Did system administrative work.\newline{}%
% Main Duties:%
%  \begin{itemize}%
%      \item Administrate the servers;
%      \item Administrate employee computers 
%          \begin{itemize}%
%              \item All employee's computers had to be up to date;
%          \end{itemize}
%      \item Did some more administrating
%   \end{itemize}}

\tlcventry{2012}{0}{Associate professor}{Universidad de Alcalá}{Madrid}{Spain}{Associate Professor.\newline{}
Main duties:%
\begin{itemize}%
 \item Teach Robotics, Intelligent Systems, Videogames Programming;
 \item Research on Evolutionary Computation and Machine Learning;
 \item Management;
\end{itemize}}

\tlcventry{2006}{2012}{Lecturer}{Universidad de Alcalá}{Alcalá de Henares, Madrid}{Spain}{Lecturer.\newline{}%
Main duties:%
\begin{itemize}%
 \item Teach Operating Systems;
 \item Research on Evolutionary Computation;
\end{itemize}}

\tlcventry{2005}{2006}{System administrator}{RedIRIS-Red.es}{Madrid}{Spain}{Stage with duties in system administration.\newline{}%
Main duties:%
\begin{itemize}%
 \item Administrate the servers (rsync, FTP, repositories);
 \item Implement a distributed search engine;
\end{itemize}}

\tlcventry{2001}{2001}{Antenna engineer}{Centre National d'Etudes Spatiales}{Toulouse}{France}{Undergraduate research stage.\newline{}%
Main duties:%
\begin{itemize}%
 \item Onboard microstrip antenna simulation;
 \item Application of finite-difference time-domain numerical analysis to antenna design;
 \item Modelization of antennas with axial symmetry;
\end{itemize}}

\tlcventry{1999}{1999}{Self-employed teacher}{}{Alcalá de Henares, Madrid}{Spain}{Teacher in courses for professional training for the unemployed.\newline{}%
Main Duties:%
\begin{itemize}%
 \item Teach web design, HTML and JavaScript;
\end{itemize}}

\subsection{Research stays}

\tlcventry{2016}{2016}{Postdoc researcher}{University of Kent}{Canterbury, UK}{}{Stage with duties in system administration.\newline{}%
Main Duties:%
\begin{itemize}%
 \item Research on Machine Learning applied to the study of randomness tests independence;
\end{itemize}}

\tlcventry{2011}{2012}{Postdoc researcher}{NASA Jet Propulsion Laboratory}{Pasadena, CA, USA}{}{Stage with duties in system administration.\newline{}%
Main Duties:%
\begin{itemize}%
 \item Research on human-multirobot interaction;
 \item Research on Brain Computer Interfaces;
\end{itemize}}

\tlcventry{2016}{2016}{Visiter researcher}{University of Portsmouth}{Postsmouth, UK}{}{Stage with duties in system administration.\newline{}%
Main Duties:%
\begin{itemize}%
 \item Research on Cryptoanalysis of lightweight authentication protocols;
 \item Research on performance assessment of Evolutionary Algorithms;
\end{itemize}}



%-------------------Skills Matrix Section----------------------------------------------------------------

\section{Skills}

% For items with categories: 
% Format:  \cvdoubleitem{Category}{List of skills}{Category Name}{List of skills}
% Note: It looks better if the category is bold with \textbf{}
% Example:
% \subsection{Development}
% \cvdoubleitem{\textbf{Languages}}{C\#, C\+\+, Java}{\textbf{Databases}}{MSSQL, MySQL}
%
% For a bullet list without categories:
% Format:  \cvlistdoubleitem{Skill 1}{Skill 2}
% Example: 
% \subsection{Development}
% \cvlistdoubleitem{C\#, Java, Ruby}{MSSQL, MySQL}
% \cvlistdoubleitem{Photoshop}{Windows, Linux. In the single column list, this item is particularly long to wrap over several lines.}

\subsection{Development}
\cvdoubleitem{\textbf{Languages}}{C, C++, Java, R, Python, JavaScript, Shell/Bash, PHP}{\textbf{Web}}{HTML/XHTML, Jquery, Ajax, Bootstrap, CSS, Jekyll, Elgg}
\cvdoubleitem{\textbf{Formats}}{XML, YAML, JSON}{\textbf{Databases}}{MySQL}
\cvdoubleitem{\textbf{VCS}}{SVN, Git}{\textbf{Tools}}{GitHub, Redmine}


\subsection{Artificial Intelligence}
\cvdoubleitem{\textbf{Optimization}}{Metaheuristics, multiobjective optimization}{\textbf{EAs}}{Genetic Algorithms, Genetic Programming, Evolution Strategies}
\cvdoubleitem{\textbf{Robotics}}{ROS}{\textbf{Tools}}{ECJ, Weka, Inspyred, PyBrain}
\cvdoubleitem{\textbf{ML}}{Artificial Neural Networks, classification, clustering, feature selection}{\textbf{}}{}

\subsection{Systems Administration}
\cvdoubleitem{\textbf{Operating Systems}}{Ubuntu, Debian, Windows}{\textbf{Web}}{Apache, Tomcat}
\cvdoubleitem{\textbf{Mail}}{Postfix, Cyrus, ClamAV}{\textbf{Virtualization}}{Virtual Box, VMWare}

\subsection{Office tools}
\cvdoubleitem{\textbf{Office}}{OpenOffice, MS Office}{\textbf{Other}}{\LaTeX, Ti\textit{k}Z, Bibtex, Markdown}

\input{cvitem_modifications/cvitem_modified}        % Removing forced punctuation from cvitem

\nocite{*}                                          % Print all publications.

\pagebreak

\section{Languages}

\cvitemwithcomment{Spanish}{Native}{}
\cvitemwithcomment{English}{Fluent}{}
\cvitemwithcomment{French}{Elemental}{}

\section{Training}

\subsection{Technical training}
\printbibliography[type=course, keyword={techtraining}, heading=none]

\subsection{Teaching training}
\printbibliography[type=course, keyword={edutraining}, heading=none]

\subsection{Other training}
\printbibliography[type=course, keyword={othertraining}, heading=none]

\section{Teaching}

\subsection{Graduate and postgraduate courses}
\printbibliography[type=course, keyword={official}, heading=none]

\subsection{Other courses}
\printbibliography[type=course, keyword={unofficial}, heading=none]

\subsection{Supervised bachelor thesis}

\printbibliography[type=bchthesis, heading=none]

\subsection{Education innovation projects}
\printbibliography[type=project, keyword={education}, heading=none]

\subsection{Teaching management}

\printbibliography[type=other, keyword={teachingmanagement}, heading=none]

\input{cvitem_modifications/cvitem_moderncvclassic} % Reverting changes to cvitem.

\subsection{Educational and cooperation stays}

\tldatecventry{2017}{Universidad Nacional Autónoma de Nicaragua - Managua}{Facultad de Ciencia e Ingeniería}{Managua, Nicaragua}{two weeks}{Initial contact to define and plan a collaboration program. Instruction in research and different topics related to AI:%
\begin{itemize}%
\item Machine Learning;
\item Research communication skills;
\end{itemize}}

\tldatecventry{2012}{Universidad Nacional Autónoma de Nicaragua - León}{Facultad de Ciencia y Tecnología}{León, Nicaragua}{two months}{Instruction in research and different topics related to AI:%
\begin{itemize}%
\item Evolutionary Algorithms;
\item Machine Learning;
\item Research communication skills;
\end{itemize}}

\tldatecventry{2010}{Bluefields Indian and Caribbean University (BICU)}{}{Bluefields, Nicaragua}{three weeks}{Teachers instruction in Operating Systems.}

\tldatecventry{2007}{Universidad Nacional Autónoma de Nicaragua - León}{Facultad de Ciencia y Tecnología}{León, Nicaragua}{two months}{Support duties for the División de Informática (IT Department) in UNAN-León.}

\tldatecventry{2006}{Universidad Nacional Autónoma de Nicaragua - León}{Facultad de Ciencia y Tecnología}{León, Nicaragua}{two months}{Teachers instruction in Operating Systems.}

\subsection{Other teaching merits}

\cvlistitem{Three ``trienios'' recognized.}

\cvlistitem{Coordinator of the teaching innovation group \textit{Inteligencia Artificial para el Aprendizaje. From 2013.}}

\cvlistitem{Member of the teaching innovation group \textit{Mentoring, Coaching y Mundos Virtuales}. From 2012 to 2013.}

\cvlistitem{Tutorship in the program Tutorías Académicas Personalizadas for students enroled in Grado de Ingeniería de Computadores, Escuela Politécnica Superior. 2014.}

\cvlistitem{Positive evaluation given by the \textit{Agencia Nacional de Evaluación de la Calidad y Acreditación (ANECA)} for the position \textit{Profesor Titular de Universidad}. 2014.}

\cvlistitem{Positive evaluation given by the \textit{Agencia Nacional de Evaluación de la Calidad y Acreditación (ANECA)} for the position \textit{Profesor Contratado Doctor}. 2013.}

\cvlistitem{Positive evaluation given by the \textit{Agencia Nacional de Evaluación de la Calidad y Acreditación (ANECA)} for the position \textit{Profesor de Universidad Privada}. 2013.}

\cvlistitem{Positive evaluation given by the \textit{Agencia Nacional de Evaluación de la Calidad y Acreditación (ANECA)} for the position \textit{Profesor Ayudante Doctor}. 2013.}

\cvlistitem{Positive evaluation given by the \textit{Agencia de Calidad, Acreditación y Prospectiva de las Universidades de Madrid (ACAP)} for the position \textit{Profesor Contratado Doctor}. 2012.}

\cvlistitem{Positive evaluation given by the \textit{Agencia de Calidad, Acreditación y Prospectiva de las Universidades de Madrid (ACAP)} for the position \textit{Profesor de Universidad Privada}. 2012.}

\cvlistitem{Positive evaluation given by the \textit{Agencia de Calidad, Acreditación y Prospectiva de las Universidades de Madrid (ACAP)} for the position \textit{Profesor Ayudante Doctor}. 2012.}

\cvlistitem{Coordinator of the Personal Tutorship Program along the course 2012/13 for Grado de Ingeniería Informática.}

\cvlistitem{Participation in the Personal Tutorship Program along the course 2013/14 for Grado de Ingeniería en Computadores.}



%-------------------Publications Section----------------------------------------------------------------
% The cvitem commands needs to be altered to correctly print all publications with the moderntime package.
% The cvitem command is edited to remove all forced punctuation within the command.
% All the typesetting of the text is handled by the modified Biblatex style.

\input{cvitem_modifications/cvitem_modified}        % Removing forced punctuation from cvitem

\nocite{*}                                          % Print all publications.

% Format:  \printbibliography[type=Biblatex type,title={Title of publication}]
% Example: \printbibliography[type=article,title={Journal Publications}]
% Example: \printbibliography[type=inproceedings,title={Conference Publications}]
% Example: \printbibliography[type=thesis,title={Thesis}]

\printbibliography[type=article,title={Journal Publications}]
\defbibfilter{conferences}{
      type=inproceedings or
      type=incollection
}
\printbibliography[filter=conferences,title={Conference Publications}]
\defbibfilter{books}{
      type=book or
      type=inbook or
      type=misc
}
\printbibliography[filter=books, title={Books and books chapters}]
\printbibliography[type=thesis,keyword={supervised},title={Supervised Thesis}]

\section{Projects}

\subsection{Competitive projects}
\printbibliography[heading=subbibliography,type=project,keyword={project},heading=none]
\subsection{Research contracts}
\printbibliography[heading=subbibliography,type=project,keyword={contract},heading=none]

\printbibliography[type=grant,title={Grants}]

\section{Research activities management}

\printbibliography[type=other, keyword=rdmanagement, heading=none]

\input{cvitem_modifications/cvitem_moderncvclassic} % Reverting changes to cvitem.

\section{Knowledge transfer}

\subsection{Patents}

\cvlistitem{\textit{Multi-Dimensional Subjective Human Interaction Proof}. US 62/329,995. 29/04/2016.}

\subsection{Entrepreneurship}

\cvlistitem{Founder and partner of \textit{Red Social NovaGob, S.L.}, a spin-off of the Universidad Autónoma de Madrid. The enterprise is dedicated to create a vertical social network for civil servants in the latin-hispanic area and provide high-level consulting to the public sector.}




%-------------------Achievements Section----------------------------------------------------------------

\section{Achievements}

\cvlistitem{Awarded with the \textbf{second prize} (\textit{accésit}) in the 8th edition of the Award for Innovative Ideas of the Universidad de Alcalá to the proposal entitled ``Lares Robotics Solutions: Sistema de teleasistencia avanzada no intrusivo''. 2015.}

\cvlistitem{\textbf{Best paper award} in the 2014 International Conference on Emerging Security Technologies Conference (EST 2014). 2014.}

\cvlistitem{Awarded with the \textbf{second prize} (\textit{accésit}) in the category of Social Sciences of the CIADE (\textit{Centro de Iniciativas Emprendedoras de la Universidad Autónoma de Madrid}) award for innovative initiatives. 2014.}

\cvlistitem{Awarded with the \textbf{second prize} of the \textit{Nuevas Aplicaciones para Internet} given by the \textit{Cátedra Telefónica para Internet de Nueva Generación} in the UPM for the proposal entitled \textit{Searchy: Un sistema distribuido para la Gestión del Conocimiento}. 2004.}

\cvlistitem{Technical coordinator of the book \textit{Construyendo la Administración Electrónica (e-Administración) local. Las tecnologías de la información y la comunicación en los ayuntamientos de la Comunidad de Madrid}, awarded with \textbf{Mención Especial (única)} in \textit{III Premio Fermín Abella y Blave para trabajos de investigación sobre la reforma administrativa en la Administración Local}, given by the \textit{Instituto Nacional de la Administración Pública}, \textit{Ministerio de Administraciones Públicas}, Spain, 2004.}


%-------------------References Section------------------------------------------------------------------

%\section{References}

% Format:  \cvreferencecolumn{\cvreference{Name Surname}{Position}{Department}{Company}{City}{Email}{Home Phone}{Cell Phone}}{\cvreference{Name Surname}{Position}{Department}{Company}{City}{Email}{Home Phone}{Cell Phone}}
% Example: 
% \subsection{Simple Solutions}
% \cvreferencecolumn{\cvreference{John Doe}{Developer}{HR}{Simple Solutions}{MyCity}{john@email.com}{+12 (34) 567 8901}{+23 (45) 678 9012}}{\cvreference{Jane Doe}{Accountant}{HR}{Simple Solutions}{MyCity}{jane@email.com}{+34 (56) 789 0123}{+45 (67) 890 1234}}
% \subsection{Monster Inc}
% \cvreferencecolumn{\cvreference{Alice Doe}{Manager}{HR}{Monster Inc}{ThatCity}{alice@email.com}{+12 (34) 567 8901}{+23 (45) 678 9012}}{}

%\subsection{Simple Solutions}
%\cvreferencecolumn{\cvreference{John Doe}{Developer}{HR}{Simple Solutions}{MyCity}{john@email.com}{+12 (34) 567 8901}{+23 (45) 678 9012}}{\cvreference{Jane Doe}{Accountant}{HR}{Simple Solutions}{MyCity}{jane@email.com}{+34 (56) 789 0123}{+45 (67) 890 1234}} \subsection{Monster Inc}
%\cvreferencecolumn{\cvreference{Alice Doe}{Manager}{HR}{Monster Inc}{ThatCity}{alice@email.com}{+12 (34) 567 8901}{+23 (45) 678 9012}}{}

%\clearpage

%-------------------Appendix----------------------------------------------------------------------------
% This section is added to append any additional documents to the cv.
% The appended documents are added to the table of contents for easier navigation of the document.
% Usage: (section)
% \phantomsection
% \addcontentsline{toc}{section}{title}
% 
% Format: (subsection)
% \phantomsection\addcontentsline{toc}{subsection}{title}
% \includepdf[pages=-]{appendix/filename.pdf}
%
% Example:
% \phantomsection
% \addcontentsline{toc}{section}{Certificates}
%
% \phantomsection
% \addcontentsline{toc}{subsection}{Landscape}
% \includepdf[pages=-]{appendix/CertificateLandscape.pdf}
%
% \phantomsection
% \addcontentsline{toc}{subsection}{Portrait}
% \includepdf[pages=-]{appendix/CertificatePortrait.pdf}

%\phantomsection
%\addcontentsline{toc}{section}{Certificates}

%\phantomsection
%\addcontentsline{toc}{subsection}{Landscape}
%\includepdf[pages=-]{appendix/CertificateLandscape.pdf}

%\phantomsection
%\addcontentsline{toc}{subsection}{Portrait}
%\includepdf[pages=-]{appendix/CertificatePortrait.pdf}

%-------------------Cover letter------------------------------------------------------------------------

%\input{coverletter.tex}                             % Include cover letter from coverletter.tex

%-------------------Document End------------------------------------------------------------------------

\end{document}

%% end of file `main.tex'.
